%%% Author: Steffen Walter  %%%
%%% Alias: firefly-serenity %%%
%%%%%%%%%%%%%%%%%%%%%%%%%%%%%%%%%%%%%%%%%%%%%%
%%%%%%%%%%%%%%%%%%%%%%%%%%%%%%%%%%%%%%%%%%%%%%
%%% DHBW Stuttgart compliant LaTeX tempalte
%%%%%%%%%%%%%%%%%%%%%%%%%%%%%%%%%%%%%%%%%%%%%%
%%%%%%%%%%%%%%%%%%%%%%%%%%%%%%%%%%%%%%%%%%%%%%

\documentclass[
a4paper,   
titlepage,  
halfparskip,
12pt        
]{scrartcl}  

%%% packages selection for different purposes (see documentation of the package maintainer for more info)
\usepackage[ngerman]{babel}
\usepackage[utf8]{inputenc}
\usepackage[T1]{fontenc}
\usepackage{ae,aecompl}
\usepackage{helvet}
\renewcommand{\familydefault}{\sfdefault}
\usepackage{amsmath,amssymb,amstext}
\usepackage{psfrag}
%\usepackage{listings}
%\lstset{language=Python} 
%\usepackage{units}
\usepackage[nottoc]{tocbibind}
\usepackage{cite}
\usepackage{caption}
\usepackage{tabto}
\usepackage{xcolor}
\usepackage{longtable}
\usepackage{lmodern}
\usepackage{setspace}
\usepackage{fancyhdr}
\usepackage{tablefootnote}
\usepackage{acronym}


%%%%%%%%%%%%%%%%%%%%%%%
%%%%%%%%%%%%%%%%%%%%%%%
%%% layout
%%%%%%%%%%%%%%%%%%%%%%%
%%%%%%%%%%%%%%%%%%%%%%%

% Enables blockset
\sloppy

\usepackage{geometry}
\geometry{a4paper, top=25mm, left=30mm, right=25mm, bottom=30mm, headsep=10mm, footskip=12mm}

%%% PDF options

\usepackage{ifpdf}

\ifpdf %if -> pdflatex
  \usepackage[pdftex]{graphicx}

  \pdfcompresslevel=9
  \usepackage[%
    pdftex=true,      
    backref,    
    pagebackref=false,
    colorlinks=false,
    hyperfootnotes=false,
    bookmarks=true,   
    bookmarksopen=false,
    bookmarksnumbered=false, 
    pdfpagemode=None   
  ]{hyperref}
  \DeclareGraphicsExtensions{.pdf}
\else %else -> latex 
  \usepackage[dvips]{graphicx}
  \DeclareGraphicsExtensions{.eps}
  \usepackage[dvips, colorlinks=false]{hyperref}
\fi

%%% PDF-Meta-Information 
\hypersetup{
  pdftitle={T1000},
  pdfauthor={Steffen Walter},
  pdfsubject={Secrets Management in großen Firmenumgebungen},
  pdfcreator={Accomplished with LaTeX2e and pdfLaTeX with hyperref-package.},
  pdfproducer={science + computing ag},
  pdfkeywords={}
}

%%%%%%%%%%%%%%%%%%%%%%%
%%%%%%%%%%%%%%%%%%%%%%%
%%% begin document
%%%%%%%%%%%%%%%%%%%%%%%
%%%%%%%%%%%%%%%%%%%%%%%

%%% header and footer before the actual body

\begin{document}

%%%%%%%%%%%%%%%%%%%%%%%
%%%%%%%%%%%%%%%%%%%%%%%
%%% titlepage
%%%%%%%%%%%%%%%%%%%%%%%
%%%%%%%%%%%%%%%%%%%%%%%

\begin{titlepage}
\begin{longtable}{lcr}
{\includegraphics[height=1.7cm]{logo}} &
{\includegraphics[height=1.05cm]{blank}} &
{\includegraphics[height=1.7cm]{dhbw}}
\end{longtable}
% \enlargethispage{20mm}
\bigskip
\bigskip
\begin{center}
\vspace*{12mm} {\LARGE\bf Secrets Management in großen Firmenumgebungen}\\
\vspace*{12mm} {\large\bf Bericht Praxis I}\\
\vspace*{3mm} {\large\bf T1000}\\
\vspace*{12mm} des Studiengangs Informationstechnik (B.Sc.)\\ an der Dualen Hochschule Baden-Württemberg Stuttgart\\
% \vspace*{3mm} an der Dualen Hochschule Baden-Württemberg\\
\vspace*{12mm} von\\
\vspace*{3mm} {\large\bf Steffen Walter}\\
\vspace*{12mm} 03.09.2018\\
\end{center}
\vfill
\begin{spacing}{1.5}
\begin{tabbing}
mmmmmmmmmmmmmmmmmmmmmmmmmm \= \kill
\textbf{Bearbeitungszeitraum} \> 4 Wochen\\
\textbf{Matrikelnummer, Kurs} \> 1145690, TINF17IN\\
\textbf{Ausbildungsunternehmen} \> science + computing ag, Tübingen\\
\textbf{Betreuer des Ausbildungsunternehmens} \>Dr. Marcus Camen\\
% \textbf{Gutachter der Hochschule} \> Bernd Beutlin\\
\end{tabbing}
\end{spacing}
\end{titlepage}

%%%%%%%%%%%%%%%%%%%%%%%
%%%%%%%%%%%%%%%%%%%%%%%
%%% Erklärung
%%%%%%%%%%%%%%%%%%%%%%%
%%%%%%%%%%%%%%%%%%%%%%%
\thispagestyle{empty}

\begin{table}[h]
\centering
  \begin{tabular}{| l |}
  \hline
  \\
  \textbf{Erklärung} \\
  \\
  Ich versichere hiermit, dass ich meine Studienarbeit mit dem Thema: \\
  ``Secrets Management in großen Firmenumgebungen`` selbstständig verfasst \\
  und keine anderen als die angegebenen Quellen und Hilfsmittel benutzt habe. \\
  \\
  Ich versichere zudem, dass die eingereichte elektronische Fassung mit der gedruckten \\
  Fassung übereinstimmt. \\ \\
  ............................................................\hspace{0.5cm} ......................................\\
  \textit{Ort} \hspace{1cm} \textit{Datum} \hspace{4.2cm} \textit{Unterschrift}\\
  \\
  \hline
  \end{tabular}
\end{table}
\newpage

%%%%%%%%%%%%%%%%%%%%%%%
%%%%%%%%%%%%%%%%%%%%%%%
%%% Abstract
%%%%%%%%%%%%%%%%%%%%%%%
%%%%%%%%%%%%%%%%%%%%%%%
\thispagestyle{empty}

\large{\textbf{Zusammenfassung}}\\
\\
Das Thema Secrets Management wird zunehmend zu einem zentralen Thema in der
elektronischen Datenverarbeitung.  Unter dem Begriff ist im Folgenden vor
allem der Umgang mit geheimen Informationen gemeint.  Geheim sind
Informationen dann, wenn es schädlich ist wenn diese Informationen
Unbefugten zugänglich sind. Beispiele für derartige Informationen sind
Passwörter, geheime Schlüssel oder geheime Dokumente.  In der
Praxisarbeit soll evaluiert werden welche Anforderungen große Unternehmen
an ihr Secrets Management stellen und welche Programme dabei häufig zum
Einsatz kommen.  In einem weiteren Schritt soll festgestellt werden welche
Anforderungen durch jene Programme nicht erfüllt werden.  Im Anschluss
soll eine vollumfängliche Secrets Management Software daraufhin untersucht
werden, in wie fern Unzulänglichkeiten von gängiger Software behoben
werden können.  Es soll auch beleuchtet werden, welche zusätzlichen
Anforderungen Cloud Umgebungen mit sich bringen.  Hierbei zu beachten ist,
welche Vorteile durch alternative Software zu erzielen sind.  Für die
Evaluierung soll eine virtuelle Umgebung erstellt werden, in welcher die
Funktionen getestet werden.
\newpage
\thispagestyle{empty}

\large{\textbf{Abstract}}\\
\\
The topic of secrets management is becoming an uprising matter in digital data processing.
In this report the term of secrets management is mainly used to describe the handling of confidential information.
Information is considered confidetial, if it can be used to to harm the owner of the secret in any way.  
Examples for information that meets this criteria would be password, private digital key or simply confidential documents.
In the paper shall be evaluated which requirements enterprises have for their secrets management and which software is currently used to try and fulfill those needs.
Subsequently the gaps between requirements and the functional range of the used Software shall be emphasised.
Furthermore a secrets managemet software with a modern apporoach shall be examined to find out whether or not this software is able to fill in the gaps of the traditional software.
Also the aspect of cloud environment shall be considered as a factor of importance. 
To achieve this task a virtual environment shall be implemented to test the features of the chosen software.


%%%%%%%%%%%%%%%%%%%%%%%
%%%%%%%%%%%%%%%%%%%%%%%
%%% dictionaries
%%%%%%%%%%%%%%%%%%%%%%%
%%%%%%%%%%%%%%%%%%%%%%%

\pagestyle{fancy}
\fancyhf{} %% remove all previous settings

\newpage
\tableofcontents
\newpage
\pagestyle{fancy}
\fancyhf{} %% clear all previous settings
\fancyhead[R]{\thepage} %% pagenumber in the upper right corner
\fancyhead[L]{VERZEICHNISSE} %% section description in the upper left corner
\pagenumbering{roman}
\section*{Abkürzungsverzeichnis} %%Title for list of acronyms
\addcontentsline{toc}{section}{Abkürzungsverzeichnis}
\begin{acronym}
 \acro{AD}{Active Directory}
 \acro{IT}{Informationstechnik}
 \acro{KRITIS}{Kritische Infrastruktur}
 \acro{OECD}{Organisation für wirtschaftliche Zusammenarbeit und Entwicklung}
\end{acronym}
\listoffigures
\listoftables
\setcounter{table}{0} %% sets tabel counter to 0 to ignore table from frontpage
\newpage
\begin{onehalfspacing} %% set space between lines to 1.5
\pagestyle{fancy}
\fancyhf{} %% remove all previous settings
%\renewcommand{\headrulewidth}{0pt} %%% activate to remove seperation line between head and main body
\fancyhead[L]{\leftmark} %% section name and number in the upper left corner
\fancyhead[R]{\thepage} %% pagenumber in upper right corner
\pagenumbering{arabic}

%%%%%%%%%%%%%%%%%%%%%%%
%%%%%%%%%%%%%%%%%%%%%%%
%%% begin main document
%%%%%%%%%%%%%%%%%%%%%%%
%%%%%%%%%%%%%%%%%%%%%%%

\section{Einleitung}
\label{sec:einleitung}

\subsection{Gegenstand und Ziele des Praxisberichts}
\label{subsec:ziele}

\subsection{Einführung in das Thema}
\label{subsec:einfuehrung}
Mit der fortschreitenden Digitalisierung nahezu aller Wirtschaftszweige steigt auch die Relevanz für die Absicherung der daraus resultierenden \ac{IT}-Infrastrukturen.
Zusätzlich zu den eigenen Sicherheitsbelangen des Betreibers einer informationstechnischen Umgebung kommen auch noch gesetzliche Regelungen wie das IT-Sicherheitsgesetz zum Tragen. 
Vor allem im Bezug auf eine zunehmende Verlagung des \ac{IT}-Betriebs hin zum Cloud-Computing und den damit verbundenen Problemen dezentraler Datenhaltung (vor allem bei den public und hybrid Modellen), entstehen häufig unübersichtliche Sicherheitskonzepte. Durch die große Varianz der Szenarien, vorallem auch in Anbetracht unterschiedlicher Sicherheitsniveaus der Daten sind die verwendeten Konzepte der unterschiedlich.\cite[S. 7f]{risiko}\newline
Spionage- und Sabotage-Angriffe werden aus den unterschiedlichsten Motivationen und auch von den Unterschielichsten Objekten verübt. So reicht das Spektrum der Angreifer von Kleinkriminellen über Geheimdienste und Terroristen bis hin zur organisierten Kriminalität. Die Aufgabe der \ac{IT}-Sicherheit besteht also darin, die potentiellen Angreifer in ihren Erwägungen zu berücksichtigen und und die Werte und Gemeimisse der Unternehmen zu schützen. Ein zentraler Faktor bei der Entwicklung eines Sicherheitskonzepts auf dieser Grundlage ist es, ein stringentes Konzept zur Kontrolle der Autorisierung einer Person oder eines Dienstes auf unterschieliche Bereiche in der zu betreuenden Umgebung. Neben der Autorisierung spielt auch die Authentifizierung, denn es muss zu jedem Zeitpunkt sichergestellt werden, dass die Autorisierung auch der richtigen Person oder Anwendung übertragen wurde.\cite[S. 9]{risiko} 
Der Aufwand welcher für \ac{IT}-Sicherheitskonzept betrieben wird bemisst sich meistens am Schaden, der zu erwarten ist, sollten geheime Daten in die Hände Dritter gelangen. Da es kaum verlässliche Daten zur Quantität der Kosten gibt, die durch einen kritischen Sicherheitsvorfall verursacht werden, ist die daran orientierte Bemessung der Sicherheitsvorkehrungen umstritten. Generell gibt es verschiedene Faktoren die bei der qualitativen Kostenabschätzung berücksichtigt werden müssen, so wird im allgemeinen zwischen direkten und indirekten Kosten unterschieden.\cite[S. 12]{kosten}
\subsubsection{Direkte Kosten}
Die direkten Kosten die durch Wirtschaftsspionage entstehen können bemessen sich zu allererst einmal an dem direkten Wert des gestohlenen Eigentums. Dieser Wert lässt sich ermitteln durch den finanziellen Gegenwert, den das Eigentum hat und an den zu erwartenden Gewinneinbußen durch den Verlust (der Exklusivität) des Eigentums. Weiterhin entstehen direkte Kosten durch die das Desaster-Recovery, das heißt durch die Schritte die eingeleitet werden müssen um den Status Quo wieder herzustellen. Zu guter Letzt werden auch noch diejenigen Kosten hinzugezählt, die durch die Prävention einer Widerholung des Ereignisses entstehen, dazu zählen Prozessänderungen, Sicherheitsunterweisungen und weitere direkte Maßnahmen.\cite[S. 13]{kosten}
\subsubsection{Indirekte Kosten}
Zu den indirekten Kosten werden vor allem die Umsatzausfälle durch Image- und Markenschäden gezählt. Außerdem entstehen hohe Umsatzeinbußen durch Plagiate welche in folge des Gestohlenen Eigentums verbreitet werden können. Plagiate können deutlich günstiger angeboten werden, da die Kosten für Forschung und Entwicklung nicht in den Preis eingerechnet werden müssen.\cite[S. 14]{kosten}
\subsubsection{Versuch einer Quntifizierung}
Nach Schätzungen der \ac{OECD} belaufen sich die Schäden durch Fälschungen und Produktpiraterie weltweit auf 638 Milliarden US-Dollar pro Jahr. Die Schätzungen diesbezüglich gehen aber weit auseinander. Es scheint jedoch Sicher zu sein, dass Sich die Schäden im dreistelligen Milliardenberich bewegen. Für die deutsche Wirtschaft liegen die Schätzungen zwischen 20 und 50 Milliarden Euro.\cite[S. 16f]{kosten}
\subsubsection{Anforderung an Sicherheitssoftware}
\label{subsubsec:anforderung}
Spezifische Anforderungen die an eine Software zum Secrets Management gestellt werden, können wie folgt zusammengefasst werden:

\begin{itemize}
  \item Auffindbarkeit - Es muss zu jedem Zeitpunkt klar sein, wo sich Secrets im Unternehmen befinden.
  \item Nachvollziebarkeit - Es muss möglich sein, Verantwortliche zu nennen.
  \item Break Glass Szenario - Es muss einen Weg geben, im Fall eines Angriffs den Schaden einzugrenzen.
  \item Verfügarkeit - Daten müssen jederzeit zugänglich sein.
  \item Integrität - Daten müssen immer vollständig und korrekt sein.
  \item Verlässlichkeit - Daten müssen authentisch sein, Kommunikationswege nachvollziehbar.
  \item Autorisierung - Der Zugriff auf Daten soll nur dann möglich sein, wenn die betroffene Person/Maschine berechtigt ist (minimale Berechtigungsvergabe).
  \item Benutzerfreundlichkeit - Einfacher Zugriff von Mensch und Maschine.
  \item Authentifizierung - Sichere Anmeldeverfahren müssen zur Verfügung stehen.
  \item Credential Management Systeme zur sicheren Erstellung, Speicherung und Änderung sollen zur verfügung stehen.
  \item Das Risiko, dass kompromittierte Verschlüsselung zum Verlust oder dem unberechtigten Offenlegen von Informationen führt, muss kontrollierbar sein.
\end{itemize}

Rechtliche Aspekte werden durch den Vorliegenden Bericht nicht weiter beleuchtet.\footnote{Die Inhalte aus \autoref{subsubsec:anforderung} wurden in Anlehung an interne Dokumenten eines großen Unternehmens und Gesprächen mit \ac{IT}-Administratoren erarbeitet}

\subsection{Stand der Technik}
\label{subsec:stand}
Um die in \autoref{subsec:einfuehrung} beschriebenen Themenkomplexe der Autorisierung und Authentifizierung in Frimenumgebungen umzusetzten gibt es verschiedene Ansätze. Im folgenden sollen einige gängige Softwareprodukte, die die genannten Aufgaben erfüllen sollen vorgestellt und auf ihre Tauglichkeit zur vollumfänglichen Erfüllung der in \autoref{subsubsec:anforderung} beschriebenen Anforderung untersucht.
\begin{table}[h]
\centering

  \begin{tabular}{|l|c|c|c|}
  \hline
  \textbf{Funktion} & \textbf{Kerberos} & \textbf{Pleasant Pass-} & \textbf{Hashicorp}\\
  Näheres: & & \textbf{word Server} & \textbf{Vault} \\
  \autoref{subsubsec:anforderung} & & (Multi-User KeePass) &\\
  \hline
  \hline
  Auffindbarkeit & \checkmark & \checkmark & \checkmark\\
  \hline
  Nachvollziebarkeit & $\times$ & \checkmark & \checkmark\\
  \hline  
  Break Glass Szenario & $\times$ & $\times$ & \checkmark\\
  \hline
  Verfügbarkeit & \checkmark & \checkmark & \checkmark\\
  \hline
  Integrität & \checkmark & \checkmark & \checkmark\\
  \hline
  Verlässlichkeit & \checkmark & ? & \checkmark\\
  \hline
  Autorisierung & $\times$ & \checkmark & \checkmark\\
  \hline
  Authentifizierung & \checkmark & \checkmark & \checkmark\\
  \hline
  Credential Management & \checkmark & \checkmark & \checkmark\\
  \hline
  Kontrollierbarkeit & $\times$ & \checkmark & \checkmark\\
  \hline
  \end{tabular}
\caption[Softwarevergleich]{Vergleich verschiedener Softwareprodukte\cite[S. 4f, 13, 57 ]{kerberos}\cite{pleasant}\cite{vault}Die Programme werden in unterschieldichen Versionen angeboten. Einige der beschriebenen Funktionen stehen möglicherweise in der kostenfreien Version nicht zur Verfügung.}
\end{table}

\subsubsection{Microsoft \acf{AD}}




\subsubsection{KeePassX}
\subsubsection{CyberArk}
\subsubsection{Hashicorp Vault}

Funktionen von Vault:

\begin{itemize}
  \item Willkürliche Verbindungen von Identifikator und Wert können auf "sichere" Art und Weise durch Vault abgelgt werden. Dabei werden die Inhalte verschlüsselt bevor sie in einen persistenten Speicher geschrieben werden.
  \item Für eine zunehmende Anzahl an Diensten kann Vault dynamische Zugangdaten generieren. Wenn zum Beispiel ein Dienst Zugriff auf eine Datenbank erhalten will, kann Vault einen (zeitlich Beschränkten) Zugriff geähren. Dabei werden temporäre Zugangsdaten (oder ein Schlüsselpaar) erstellt, welche durch Vault nach Ablauf der Gültigkeit widerrufen werden.
  \item Daten welche sensible Inhalte haben, können durch Vault verschlüsselt werden, ohne dass sie durch Vault im eigenen Backend gespeichert werden müssen. Entwicker sind damit in der Lage Daten durch Vault verschlüsseln zu lassen um im Anschluss nach Belieben weiterzuverarbeiten.
  \item Alle Geheimnisse welche durch Vault gespeichert werden haben eine Gültigkeitsdauer. Nach Ablauf der zugeordneten Gültikeitsdauer werden die betroffenen Geheimnisse durch Vault widerrufen. Clients können durch entsprechende APIs die Gültikeit eines Geheimnisses verlängern bzw. erneuern.
  \item Geheimnisse können auch durch Administratoren widerrufen werden. Dabei bietet Vault die Funktion, dass bei Bedarf ganze Baumstrukturen an Geheimnissen auf einmal ihre Gültikeit verlieren. Es kann auf unterschiedliche Weisen gefiltert werden, so können zum Beispiel alle Geheimnisse widerrufen werden auf die ein spezieller User zugegriffen hat.

Die Komponenten von Vault:
  \item Storage Backend: Vault benötigt ein Storage Backend um verschlüsselte Daten abzulegen. Die einzige Anforderung an das Storage Backend ist, dass es möglichst strapazierfähig ist. Vault vertraut dem Storage Backend nicht und es wird nicht davon ausgegangen, dass es speziell gegen fremde Zugriffe geschützt ist. 
  \item Barrier: Zwischen Storage und Vault wird jede Kommunikation durch eine Art Kontrollpunkt geprüft. Durch diesen Mechanismus soll sichergestellt werden, dass alle Daten welche von Vault in Richtung Storage Backend übermittelt werden, zwansläufig verschlüsslt werden. Außerdem ist der Mechanismus dafür zuständig alle Daten die aus dem Storage Backend gelesen werden zu verifizieren und zu entschlüsseln. 
  \item Secret Backend: Das Secret Backend ist dafür Zuständig auf Anfrage ein angefordertes Geheimnis preiszugeben. Bei einigen Systemen ist der Prozess statisch organisiert. Das bedeutet, dass bei der gleichen Anfrage immer die gleiche Rückgabe zu erwarten ist. Andere Systeme arbeiten hier etwas komplexer, sie sind dazu in der Lage dynamische Zugangsdaten zu erzeugen. Diese Möglichkeit schafft eine zusätzliche Sicherheitsebene. Das Feature steht allerdings nicht für jede Anwendung zur Verfügung
  \item Client Token: Ein Client Token wird ausgestellt, um einen Client über die dauer einer Sitzung gegenüber Vault zu authentisieren.
  \item Server: Vault wird als einzelne Binärdatei zur Verfügung gestellt, es kann sowohl als Client als auch als Server ausgeführt werden. Wenn der Server gestartet wurde, kümmert er sich um die Kommunikation mit den Backends und stellt ein API für die Clientinteraktion bereit. Außerdem ist er verantwortlich für die Anwendung der ACLs und den Widerruf abgelaufener Geheimnisse. Neben einigen weiteren Aufgaben erstellt der Server auch ein Log in welchem jede Interaktion mit Vault dokumentiert wird. 
\end{itemize}






Motivation der Aufgabenstellung
Vorstellung Projektumgebung
Vorausblick

\section{Hauptteil}
\label{sec:hauptteil}

Teil 1
Theoretische Ausarbeitung als Vorarbeit zur Durchführung
des Projekts

Teil 2
Anforderungsdefinition
Anforderungsanalyse
Lösungsgenerierung
Lösungsbewertung
Umsetzung

\section{Zusammenfassung und Ausblick}
\label{sec:ausblick}

%%%%%%%%%%%%%%%%%%%%%%%
%%%%%%%%%%%%%%%%%%%%%%%
%%% glossary
%%%%%%%%%%%%%%%%%%%%%%%
%%%%%%%%%%%%%%%%%%%%%%%
\newpage

\pagestyle{fancy}
\fancyhf{} %% clear all previous settings
\fancyhead[R]{\thepage} %% pagenumber in the upper right corner
\fancyhead[L]{GLOSSAR} %% section description in the upper left corner

\section*{Glossar} 
\addcontentsline{toc}{section}{Glossar}
\begin{acronym}
 \acro{Authentifizierung}{der Identitätsnachweis einer Person, Maschine oder eines Dienstes, gegenüber einer weiteren Instanz} 
 \acro{Autorisierung}{die explizite Freigabe um auf einen geheimen Inhalt zuzugreifen}
 \acro{Cloud-Computing} {die Verwendung scheinbar  unendlicher IT-Ressourcen,  die bedarfsgerecht und flexibel zur Verfügung gestellt werden können. Clouds können in unterschielichen Formen betrieben werden, so gibt es sogenannte private, public und hybrid-Clouds. Sie unterscheiden sich darin ob die Clouddienste auf eigener Infrastruktur, bei einem Cloudhoster oder gemischt betrieben werden.\cite[S. 3]{cloud}}
 \acro{Dienst}{eine autarke Einheit, welche eine spezifizierte Aufgabe bzw. Funktionalitaet erfuellt und diese ueber keine klar definierte Schnittstelle zur Verfuegung stellt}
\end{acronym}

\newpage
\pagestyle{fancy}
\fancyhf{} %% clear all previous settings
\fancyhead[R]{\thepage} %% pagenumber in the upper right corner
\fancyhead[L]{\leftmark} %% section description in the upper left corner
%%%%%%%%%%%%%%%%%%%%%%%
%%%%%%%%%%%%%%%%%%%%%%%
%%% end main document
%%%%%%%%%%%%%%%%%%%%%%%
%%%%%%%%%%%%%%%%%%%%%%%

\appendix
\bibliographystyle{plain}
\bibliography{literatur}

\end{onehalfspacing}
\end{document}

%%%%%%%%%%%%%%%%%%%%%%%%%%%%%%%%%%%%%%%%%%%%%%
%%%%%%%%%%%%%%%%%%%%%%%%%%%%%%%%%%%%%%%%%%%%%%
%%%%%%%%%%%%%%%%%%%%%%%%%%%%%%%%%%%%%%%%%%%%%%
%%%%%%%%%%%%%%%%%%%%%%%%%%%%%%%%%%%%%%%%%%%%%%

