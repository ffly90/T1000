%%% Author: Steffen Walter  %%%
%%% Alias: firefly-serenity %%%
%%%%%%%%%%%%%%%%%%%%%%%%%%%%%%%%%%%%%%%%%%%%%%
%%%%%%%%%%%%%%%%%%%%%%%%%%%%%%%%%%%%%%%%%%%%%%
%%% DHBW Stuttgart compliant LaTeX tempalte
%%%%%%%%%%%%%%%%%%%%%%%%%%%%%%%%%%%%%%%%%%%%%%
%%%%%%%%%%%%%%%%%%%%%%%%%%%%%%%%%%%%%%%%%%%%%%

\documentclass[
a4paper,   
titlepage,  
halfparskip,
12pt        
]{scrartcl}  

%%% packages selection for different purposes (see documentation of the package maintainer for more info)
\usepackage[ngerman]{babel}
\usepackage[utf8]{inputenc}
\usepackage[T1]{fontenc}
\usepackage{ae,aecompl}
\usepackage{helvet}
\renewcommand{\familydefault}{\sfdefault}
%\usepackage{amsmath,amssymb,amstext}
\usepackage{psfrag}
%\usepackage{listings}
%\lstset{language=Python} 
%\usepackage{units}
\usepackage[nottoc]{tocbibind}
\usepackage{cite}
\usepackage{caption}
\usepackage{tabto}
\usepackage{xcolor}
\usepackage{longtable}
\usepackage{lmodern}
\usepackage{setspace}
\usepackage{fancyhdr}
\usepackage{tablefootnote}
\usepackage[printonlyused]{acronym}

%%%%%%%%%%%%%%%%%%%%%%%
%%%%%%%%%%%%%%%%%%%%%%%
%%% layout
%%%%%%%%%%%%%%%%%%%%%%%
%%%%%%%%%%%%%%%%%%%%%%%

% Enables blockset
\sloppy

\usepackage{geometry}
\geometry{a4paper, top=25mm, left=30mm, right=25mm, bottom=30mm, headsep=10mm, footskip=12mm}

%%% PDF options

\usepackage{ifpdf}

\ifpdf %if -> pdflatex
  \usepackage[pdftex]{graphicx}

  \pdfcompresslevel=9
  \usepackage[%
    pdftex=true,      
    backref,    
    pagebackref=false,
    colorlinks=false,
    hyperfootnotes=false,
    bookmarks=true,   
    bookmarksopen=false,
    bookmarksnumbered=false, 
    pdfpagemode=None   
  ]{hyperref}
  \DeclareGraphicsExtensions{.pdf}
\else %else -> latex 
  \usepackage[dvips]{graphicx}
  \DeclareGraphicsExtensions{.eps}
  \usepackage[dvips, colorlinks=false]{hyperref}
\fi

%%% PDF-Meta-Information 
\hypersetup{
  pdftitle={T1000},
  pdfauthor={Steffen Walter},
  pdfsubject={Secrets Management in großen Firmenumgebungen},
  pdfcreator={Accomplished with LaTeX2e and pdfLaTeX with hyperref-package.},
  pdfproducer={science + computing ag},
  pdfkeywords={}
}

%%%%%%%%%%%%%%%%%%%%%%%
%%%%%%%%%%%%%%%%%%%%%%%
%%% begin document
%%%%%%%%%%%%%%%%%%%%%%%
%%%%%%%%%%%%%%%%%%%%%%%

%%% header and footer before the actual body

\begin{document}

%%%%%%%%%%%%%%%%%%%%%%%
%%%%%%%%%%%%%%%%%%%%%%%
%%% titlepage
%%%%%%%%%%%%%%%%%%%%%%%
%%%%%%%%%%%%%%%%%%%%%%%

\begin{titlepage}
\begin{longtable}{lcr}
{\includegraphics[height=1.7cm]{logo}} &
{\includegraphics[height=1.05cm]{blank}} &
{\includegraphics[height=1.7cm]{dhbw}}
\end{longtable}
% \enlargethispage{20mm}
\bigskip
\bigskip
\begin{center}
\vspace*{12mm} {\LARGE\bf Secrets Management in großen Firmenumgebungen}\\
\vspace*{12mm} {\large\bf Bericht Praxis I}\\
\vspace*{3mm} {\large\bf T1000}\\
\vspace*{12mm} des Studiengangs Informationstechnik (B.Sc.)\\ an der Dualen Hochschule Baden-Württemberg Stuttgart\\
% \vspace*{3mm} an der Dualen Hochschule Baden-Württemberg\\
\vspace*{12mm} von\\
\vspace*{3mm} {\large\bf Steffen Walter}\\
\vspace*{12mm} 30.08.2018\\
\end{center}
\vfill
\begin{spacing}{1.5}
\begin{tabbing}
mmmmmmmmmmmmmmmmmmmmmmmmmm \= \kill
\textbf{Bearbeitungszeitraum} \> 4 Wochen\\
\textbf{Matrikelnummer, Kurs} \> 1145690, TINF17IN\\
\textbf{Ausbildungsunternehmen} \> science + computing ag, Tübingen\\
\textbf{Betreuer des Ausbildungsunternehmens} \>Dr. Marcus Camen\\
% \textbf{Gutachter der Hochschule} \> Bernd Beutlin\\
\end{tabbing}
\end{spacing}
\end{titlepage}

%%%%%%%%%%%%%%%%%%%%%%%
%%%%%%%%%%%%%%%%%%%%%%%
%%% Erklärung
%%%%%%%%%%%%%%%%%%%%%%%
%%%%%%%%%%%%%%%%%%%%%%%
\thispagestyle{empty}

\begin{table}[h]
\centering
  \begin{tabular}{| l |}
  \hline
  \\
  \textbf{Erklärung} \\
  \\
  Ich versichere hiermit, dass ich meine Studienarbeit mit dem Thema: \\
  ``Secrets Management in großen Firmenumgebungen`` selbstständig verfasst \\
  und keine anderen als die angegebenen Quellen und Hilfsmittel benutzt habe. \\
  \\
  Ich versichere zudem, dass die eingereichte elektronische Fassung mit der gedruckten \\
  Fassung übereinstimmt. \\ \\
  ............................................................\hspace{0.5cm} ......................................\\
  \textit{Ort} \hspace{1cm} \textit{Datum} \hspace{4.2cm} \textit{Unterschrift}\\
  \\
  \hline
  \end{tabular}
\end{table}
\newpage

%%%%%%%%%%%%%%%%%%%%%%%
%%%%%%%%%%%%%%%%%%%%%%%
%%% Abstract
%%%%%%%%%%%%%%%%%%%%%%%
%%%%%%%%%%%%%%%%%%%%%%%
\thispagestyle{empty}

\large{\textbf{Zusammenfassung}}\\
\\
Das Thema Secrets Management wird zunehmend zu einem zentralen Thema in der
elektronischen Datenverarbeitung.  Unter dem Begriff ist im Folgenden vor
allem der Umgang mit geheimen Informationen gemeint.  Geheim sind
Informationen dann, wenn es schädlich ist wenn diese Informationen
Unbefugten zugänglich sind. Beispiele für derartige Informationen sind
Passwörter, geheime Schlüssel oder geheime Dokumente.  In der
Praxisarbeit soll evaluiert werden welche Anforderungen große Unternehmen
an ihr Secrets Management stellen und welche Programme dabei häufig zum
Einsatz kommen.  In einem weiteren Schritt soll festgestellt werden welche
Anforderungen durch jene Programme, nicht erfüllt werden.  Im Anschluss
soll eine vollumfängliche Secrets Management Software daraufhin untersucht
werden, in wie fern Unzulänglichkeiten von gängiger Software behoben
werden können.  Es soll auch beleuchtet werden, welche zusätzlichen
Anforderungen Cloud Umgebungen mit sich bringen.  Hierbei zu beachten ist,
welche Vorteile durch alternative Software zu erzielen sind.  Für die
Evaluierung soll eine virtuelle Umgebung erstellt werden, in welcher die
Funktionen getestet werden.
\newpage
\thispagestyle{empty}

\large{\textbf{Abstract}}

%%%%%%%%%%%%%%%%%%%%%%%
%%%%%%%%%%%%%%%%%%%%%%%
%%% dictionaries
%%%%%%%%%%%%%%%%%%%%%%%
%%%%%%%%%%%%%%%%%%%%%%%

\pagestyle{fancy}
\fancyhf{} %% remove all previous settings

\newpage
\tableofcontents
\newpage
\pagestyle{fancy}
\fancyhf{} %% clear all previous settings
\fancyhead[R]{\thepage} %% pagenumber in the upper right corner
\fancyhead[L]{VERZEICHNISSE} %% section description in the upper left corner
\pagenumbering{roman}
\section*{Abkürzungsverzeichnis} %%Title for list of acronyms
\addcontentsline{toc}{section}{Abkürzungsverzeichnis}
\begin{acronym}
 \acro{IT}{Informationstechnik}
 \acro{KRITIS}{Kritische Infrastruktur}
 \acro{OECD}{Organisation für wirtschaftliche Zusammenarbeit und Entwicklung}
\end{acronym}
\listoffigures
\listoftables
\setcounter{table}{0} %% sets tabel counter to 0 to ignore table from frontpage
\newpage
\begin{onehalfspacing} %% set space between lines to 1.5
\pagestyle{fancy}
\fancyhf{} %% remove all previous settings
%\renewcommand{\headrulewidth}{0pt} %%% activate to remove seperation line between head and main body
\fancyhead[L]{\leftmark} %% section name and number in the upper left corner
\fancyhead[R]{\thepage} %% pagenumber in upper right corner
\pagenumbering{arabic}

%%%%%%%%%%%%%%%%%%%%%%%
%%%%%%%%%%%%%%%%%%%%%%%
%%% begin main document
%%%%%%%%%%%%%%%%%%%%%%%
%%%%%%%%%%%%%%%%%%%%%%%

\section{Einleitung}
\label{sec:einleitung}

\subsection{Gegenstand und Ziele des Praxisberichts}
\label{subsec:ziele}

\subsection{Einführung in das Thema}
\label{subsec:einführung}
Mit der fortschreitenden Digitalisierung nahezu aller Wirtschaftszweige steigt auch die Relevanz für die Absicherung der daraus resultierenden \ac{IT}-Infrastrukturen.
Zusätzlich zu den eigenen Sicherheitsbelangen des Betreibers einer informationstechnischen Umgebung kommen auch noch gesetzliche Regelungen wie das IT-Sicherheitsgesetz zum Tragen. 
Vor allem im Bezug auf eine zunehmende Verlagung des \ac{IT}-Betriebs hin zum Cloud-Computing\footnote{Unter Cloud-Computing versteht man die Verwendung scheinbar  unendlicher  
IT-Ressourcen,  die bedarfsgerecht und flexibel zur Verfügung gestellt werden können. Clouds können in unterschielichen Formen betrieben werden, so gibt es sogenannte private, public und hybrid-Clouds. Sie unterscheiden sich darin ob die Clouddienste auf eigener Infrastruktur, bei einem Cloudhoster oder gemischt betrieben werden.\cite[S. 3]{cloud}} und den damit verbundenen Problemen dezentraler Datenhaltung (vor allem bei den public und hybrid Modellen), entstehen häufig unübersichtliche Sicherheitskonzepte. Durch die große Varianz der Szenarien, vorallem auch in Anbetracht unterschiedlicher Sicherheitsniveaus der Daten sind die verwendeten Konzepte der unterschiedlich.\cite[S. 7f]{risiko}\newline
Spionage- und Sabotage-Angriffe werden aus den unterschiedlichsten Motivationen und auch von den Unterschielichsten Objekten verübt. So reicht das Spektrum der Angreifer von Kleinkriminellen über Geheimdienste und Terroristen bis hin zur organisierten Kriminalität. Die Aufgabe der \ac{IT}-Sicherheit besteht also darin, die potentiellen Angreifer in ihren Erwägungen zu berücksichtigen und und die Werte und Gemeimisse der Unternehmen zu schützen. Ein zentraler Faktor bei der Entwicklung eines Sicherheitskonzepts auf dieser Grundlage ist es, ein stringentes Konzept zur Kontrolle der Authorisierung\footnote{Zugangsberechtigung} einer Person oder eines Dienstes\footnote{Ein Dienst ist eine autarke Einheit, welche eine spezifierte Aufgabe erfüllt bzw. Funktionaltät zur Verfügung stellt und diese über keine klar definierte Schnittstelle zur Verfügung stellt} auf unterschieliche Bereiche in der zu betreuenden Umgebung. 

Der Aufwand welcher für \ac{IT}-Sicherheitskonzept betrieben wird bemisst sich meistens am Schaden, der zu erwarten ist, sollten geheime Daten in die Hände Dritter gelangen. Da es kaum verlässliche Daten zur Quantität der Kosten gibt, die durch einen kritischen Sicherheitsvorfall verursacht werden, ist die daran orientierte Bemessung der Sicherheitsvorkehrungen umstritten. Generell gibt es verschiedene Faktoren die bei der qualitativen Kostenabschätzung berücksichtigt werden müssen, so wird im allgemeinen zwischen direkten und indirekten Kosten unterschieden.\cite[S. 12]{kosten}
\subsubsection{Direkte Kosten}
Die direkten Kosten die durch Wirtschaftsspionage entstehen können bemessen sich zu allererst einmal an dem direkten Wert des gestohlenen Eigentums. Dieser Wert lässt sich ermitteln durch den finanziellen Gegenwert, den das Eigentum hat und an den zu erwartenden Gewinneinbußen durch den Verlust (der Exklusivität) des Eigentums. Weiterhin entstehen direkte Kosten durch die das Desaster-Recovery, das heißt durch die Schritte die eingeleitet werden müssen um den Status Quo wieder herzustellen. Zu guter Letzt werden auch noch diejenigen Kosten hinzugezählt, die durch die Prävention einer Widerholung des Ereignisses entstehen, dazu zählen Prozessänderungen, Sicherheitsunterweisungen und weitere direkte Maßnahmen.\cite[S. 13]{kosten}
\subsubsection{Indirekte Kosten}
Zu den indirekten Kosten werden vor allem die Umsatzausfälle durch Image- und Markenschäden gezählt. Außerdem entstehen hohe Umsatzeinbußen durch Plagiate welche in folge des Gestohlenen Eigentums verbreitet werden können. Plagiate können deutlich günstiger angeboten werden, da die Kosten für Forschung und Entwicklung nicht in den Preis eingerechnet werden müssen.\cite[S. 14]{kosten}
\subsubsection{Versuch einer Quntifizierung}
Nach Schätzungen der \ac{OECD} belaufen sich die Schäden durch Fälschungen und Produktpiraterie weltweit auf 638 Milliarden US-Dollar pro Jahr. Die Schätzungen diesbezüglich gehen aber weit auseinander. Es scheint jedoch Sicher zu sein, dass Sich die Schäden im dreistelligen Milliardenberich bewegen. Für die deutsche Wirtschaft liegen die Schätzungen zwischen 20 und 50 Milliarden Euro.\cite[S. 16f]{kosten}

\subsection{Stand der Technik}
\label{subsec:stand}


Motivation der Aufgabenstellung
Vorstellung Projektumgebung
Vorausblick

\section{Hauptteil}
\label{sec:Hauptteil}

Teil 1
Theoretische Ausarbeitung als Vorarbeit zur Durchführung
des Projekts

Teil 2
Anforderungsdefinition
Anforderungsanalyse
Lösungsgenerierung
Lösungsbewertung
Umsetzung

\section{Zusammenfassung und Ausblick}
\label{sec:ausblick}



%%%%%%%%%%%%%%%%%%%%%%%
%%%%%%%%%%%%%%%%%%%%%%%
%%% end main document
%%%%%%%%%%%%%%%%%%%%%%%
%%%%%%%%%%%%%%%%%%%%%%%

\appendix
\bibliographystyle{plain}
\newpage
\bibliography{literatur}

\end{onehalfspacing}
\end{document}

%%%%%%%%%%%%%%%%%%%%%%%%%%%%%%%%%%%%%%%%%%%%%%
%%%%%%%%%%%%%%%%%%%%%%%%%%%%%%%%%%%%%%%%%%%%%%
%%%%%%%%%%%%%%%%%%%%%%%%%%%%%%%%%%%%%%%%%%%%%%
%%%%%%%%%%%%%%%%%%%%%%%%%%%%%%%%%%%%%%%%%%%%%%

