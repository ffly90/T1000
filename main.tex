%%% Author: Steffen Walter  %%%
%%% Alias: firefly-serenity %%%
%%%%%%%%%%%%%%%%%%%%%%%%%%%%%%%%%%%%%%%%%%%%%%
%%%%%%%%%%%%%%%%%%%%%%%%%%%%%%%%%%%%%%%%%%%%%%
%%% DHBW Stuttgart compliant LaTeX tempalte
%%%%%%%%%%%%%%%%%%%%%%%%%%%%%%%%%%%%%%%%%%%%%%
%%%%%%%%%%%%%%%%%%%%%%%%%%%%%%%%%%%%%%%%%%%%%%

\documentclass[
a4paper,   
titlepage,  
halfparskip,
12pt        
]{scrartcl}  

%%% packages selection for different purposes (see documentation of the package maintainer for more info)
\usepackage[ngerman]{babel}
\usepackage[utf8]{inputenc}
\usepackage[T1]{fontenc}
\usepackage{ae,aecompl}
\usepackage{helvet}
\renewcommand{\familydefault}{\sfdefault}
%\usepackage{amsmath,amssymb,amstext}
\usepackage{psfrag}
%\usepackage{listings}
%\lstset{language=Python} 
%\usepackage{units}
\usepackage[nottoc]{tocbibind}
\usepackage{cite}
\usepackage{caption}
\usepackage{tabto}
\usepackage{xcolor}
\usepackage{longtable}
\usepackage{lmodern}
\usepackage{setspace}
\usepackage{fancyhdr}
\usepackage{tablefootnote}
\usepackage[printonlyused]{acronym}

%%%%%%%%%%%%%%%%%%%%%%%
%%%%%%%%%%%%%%%%%%%%%%%
%%% layout
%%%%%%%%%%%%%%%%%%%%%%%
%%%%%%%%%%%%%%%%%%%%%%%

% Enables blockset
\sloppy

\usepackage{geometry}
\geometry{a4paper, top=25mm, left=30mm, right=25mm, bottom=30mm, headsep=10mm, footskip=12mm}

%%% PDF options

\usepackage{ifpdf}

\ifpdf %if -> pdflatex
  \usepackage[pdftex]{graphicx}

  \pdfcompresslevel=9
  \usepackage[%
    pdftex=true,      
    backref,    
    pagebackref=false,
    colorlinks=false,
    hyperfootnotes=false,
    bookmarks=true,   
    bookmarksopen=false,
    bookmarksnumbered=false, 
    pdfpagemode=None   
  ]{hyperref}
  \DeclareGraphicsExtensions{.pdf}
\else %else -> latex 
  \usepackage[dvips]{graphicx}
  \DeclareGraphicsExtensions{.eps}
  \usepackage[dvips, colorlinks=false]{hyperref}
\fi

%%% PDF-Meta-Information 
\hypersetup{
  pdftitle={T1000},
  pdfauthor={Steffen Walter},
  pdfsubject={Secrets Management in großen Firmenumgebungen},
  pdfcreator={Accomplished with LaTeX2e and pdfLaTeX with hyperref-package.},
  pdfproducer={science + computing ag},
  pdfkeywords={}
}

%%%%%%%%%%%%%%%%%%%%%%%
%%%%%%%%%%%%%%%%%%%%%%%
%%% begin document
%%%%%%%%%%%%%%%%%%%%%%%
%%%%%%%%%%%%%%%%%%%%%%%

%%% header and footer before the actual body

\begin{document}

%%%%%%%%%%%%%%%%%%%%%%%
%%%%%%%%%%%%%%%%%%%%%%%
%%% titlepage
%%%%%%%%%%%%%%%%%%%%%%%
%%%%%%%%%%%%%%%%%%%%%%%

\begin{titlepage}
\begin{longtable}{lcr}
{\includegraphics[height=1.7cm]{logo}} &
{\includegraphics[height=1.05cm]{blank}} &
{\includegraphics[height=1.7cm]{dhbw}}
\end{longtable}
% \enlargethispage{20mm}
\bigskip
\bigskip
\begin{center}
\vspace*{12mm} {\LARGE\bf Secrets Management in großen Firmenumgebungen}\\
\vspace*{12mm} {\large\bf Bericht Praxis I}\\
\vspace*{3mm} {\large\bf T1000}\\
\vspace*{12mm} des Studiengangs Informationstechnik (B.Sc.)\\ an der Dualen Hochschule Baden-Württemberg Stuttgart\\
% \vspace*{3mm} an der Dualen Hochschule Baden-Württemberg\\
\vspace*{12mm} von\\
\vspace*{3mm} {\large\bf Steffen Walter}\\
\vspace*{12mm} 30.08.2018\\
\end{center}
\vfill
\begin{spacing}{1.5}
\begin{tabbing}
mmmmmmmmmmmmmmmmmmmmmmmmmm \= \kill
\textbf{Bearbeitungszeitraum} \> 4 Wochen\\
\textbf{Matrikelnummer, Kurs} \> 1145690, TINF17IN\\
\textbf{Ausbildungsunternehmen} \> science + computing ag, Tübingen\\
\textbf{Betreuer des Ausbildungsunternehmens} \>Dr. Marcus Camen\\
% \textbf{Gutachter der Hochschule} \> Bernd Beutlin\\
\end{tabbing}
\end{spacing}
\end{titlepage}

%%%%%%%%%%%%%%%%%%%%%%%
%%%%%%%%%%%%%%%%%%%%%%%
%%% Erklärung
%%%%%%%%%%%%%%%%%%%%%%%
%%%%%%%%%%%%%%%%%%%%%%%
\thispagestyle{empty}

\begin{table}[h]
\centering
  \begin{tabular}{| l |}
  \hline
  \\
  \textbf{Erklärung} \\
  \\
  Ich versichere hiermit, dass ich meine Studienarbeit mit dem Thema: \\
  ``Secrets Management in großen Firmenumgebungen`` selbstständig verfasst \\
  und keine anderen als die angegebenen Quellen und Hilfsmittel benutzt habe. \\
  \\
  Ich versichere zudem, dass die eingereichte elektronische Fassung mit der gedruckten \\
  Fassung übereinstimmt. \\ \\
  ............................................................\hspace{0.5cm} ......................................\\
  \textit{Ort} \hspace{1cm} \textit{Datum} \hspace{4.2cm} \textit{Unterschrift}\\
  \\
  \hline
  \end{tabular}
\end{table}
\newpage

%%%%%%%%%%%%%%%%%%%%%%%
%%%%%%%%%%%%%%%%%%%%%%%
%%% Abstract
%%%%%%%%%%%%%%%%%%%%%%%
%%%%%%%%%%%%%%%%%%%%%%%
\thispagestyle{empty}

\large{\textbf{Zusammenfassung}}\\
\\
Das Thema Secrets Management wird zunehmend zu einem zentralen Thema in der elektronischen Datenverarbeitung. Unter dem Begriff des Secrets Management ist im folgenden vor allem der Umgang mit geheimen Informationen gemeint. Geheim sind Informationen dann, wenn es schädlich ist wenn diese Informationen Unbefugten zugänglich sind. Beispiele für derartige geheime Informationen sind Passwörter, geheime Schlüssel oder geheime Dokumente. In der Praxisarbeit soll evaluliert werden welche Anforderungen große Unternehmen an ihr Secrets Management stellen und welche Programme dabei häufig zum Einsatz kommen. In einem weiteren Schritt soll festgestellt werden welche Anforderungen durch jene Programme, nicht erfüllt werden. Im Anschluss soll eine vollumfängliche Secrets Management Software daraufhin untersucht werden, in wie fern unzulänglichkeiten von gängiger Software in diesem Bereich behoben werden können. Es soll auf beleuchtet werden welche zusätzlichen Anforderungen die Arbeit mit Cloud Umgebungen zu beachten sind welche Vorteile auch hier durch alternative Software zu erziehlen sind. Dazu soll eine virtuelle Umgebung erstellt werden um die Funktionen zu testen. 
Nach erfolgreichem Test soll ein Plan erstellt werden, wie ein Umzug von gängiger Software auf ein vollumfägliches Secrets Management System aussehen kann und es soll festgestellt werden was bei einer solgen Umstellung sinnvollerweise zu beachten ist.
\newpage
\thispagestyle{empty}

\centering{\large{\textbf{Abstract}}}

%%%%%%%%%%%%%%%%%%%%%%%
%%%%%%%%%%%%%%%%%%%%%%%
%%% dictionaries
%%%%%%%%%%%%%%%%%%%%%%%
%%%%%%%%%%%%%%%%%%%%%%%

\pagestyle{fancy}
\fancyhf{} %% remove all previous settings

\newpage
\tableofcontents
\newpage
\pagestyle{fancy}
\fancyhf{} %% clear all previous settings
\fancyhead[R]{\thepage} %% pagenumber in the upper right corner
\fancyhead[L]{VERZEICHNISSE} %% section description in the upper left corner
\pagenumbering{roman}
\section*{Abkürzungsverzeichnis} %%Title for list of acronyms
\addcontentsline{toc}{section}{Abkürzungsverzeichnis}
\begin{acronym}
 \acro{PC}{Personal Computer}
\end{acronym}
\listoffigures
\listoftables
\setcounter{table}{0} %% sets tabel counter to 0 to ignore table from frontpage
\newpage
\begin{onehalfspacing} %% set space between lines to 1.5
\pagestyle{fancy}
\fancyhf{} %% remove all previous settings
%\renewcommand{\headrulewidth}{0pt} %%% activate to remove seperation line between head and main body
\fancyhead[L]{\leftmark} %% section name and number in the upper left corner
\fancyhead[R]{\thepage} %% pagenumber in upper right corner
\pagenumbering{arabic}

%%%%%%%%%%%%%%%%%%%%%%%
%%%%%%%%%%%%%%%%%%%%%%%
%%% begin main document
%%%%%%%%%%%%%%%%%%%%%%%
%%%%%%%%%%%%%%%%%%%%%%%

\section{Einleitung}
\label{sec:einleitung}

Gegenstand und Ziele der Arbeit / Aufgabenbeschreibung
Einführung in das Thema
Stand der Technik / Forschung
Motivation der Aufgabenstellung
Vorstellung Projektumgebung
Vorausblick

\section{Hauptteil}
\label{sec:Hauptteil}

Teil 1
Theoretische Ausarbeitung als Vorarbeit zur Durchführung
des Projekts

Teil 2
Anforderungsdefinition
Anforderungsanalyse
Lösungsgenerierung
Lösungsbewertung
Umsetzung

\section{Zusammenfassung und Ausblick}
\label{sec:ausblick}



%%%%%%%%%%%%%%%%%%%%%%%
%%%%%%%%%%%%%%%%%%%%%%%
%%% end main document
%%%%%%%%%%%%%%%%%%%%%%%
%%%%%%%%%%%%%%%%%%%%%%%

\appendix
\bibliographystyle{plain}
\newpage
\bibliography{literatur}

\end{onehalfspacing}
\end{document}

%%%%%%%%%%%%%%%%%%%%%%%%%%%%%%%%%%%%%%%%%%%%%%
%%%%%%%%%%%%%%%%%%%%%%%%%%%%%%%%%%%%%%%%%%%%%%
%%%%%%%%%%%%%%%%%%%%%%%%%%%%%%%%%%%%%%%%%%%%%%
%%%%%%%%%%%%%%%%%%%%%%%%%%%%%%%%%%%%%%%%%%%%%%

